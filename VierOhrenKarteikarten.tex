% LaTeX-Vorlage für Vier-Ohren-Karteikarten
% Siehe Anleitung im Download-Bereich der App
\documentclass[a4paper,10pt]{article}
\usepackage[utf8]{inputenc}
\usepackage[T1]{fontenc}
\usepackage[german]{babel}
\usepackage{geometry}
\geometry{a4paper, margin=1cm}
\usepackage{tcolorbox}
\tcbuselibrary{breakable}
\usepackage{xcolor}
\usepackage{enumitem}

\definecolor{calmblue}{RGB}{100,149,237}
\definecolor{calmgrey}{RGB}{220,220,220}

\begin{document}

\begin{tcolorbox}[colback=calmgrey, colframe=calmblue, title={\textbf{Vier-Ohren-Modell: Karteikarten für unser Superteam}}, breakable]
\textbf{Wir sind zwei Helden}, die gemeinsam Missverständnisse klären und unsere Kommunikation stärken. Diese Karteikarten helfen uns, den Alltag zu meistern, Trigger zu erkennen und Vorwürfe zu vermeiden. Jede Karte beschreibt eine Ebene des Vier-Ohren-Modells (Schulz von Thun) mit Übungen und Beispielen. \newline
\textit{Stell dir vor}: Unsere Kommunikation ist wie ein Radio mit vier Kanälen. Mit diesen Karten stellen wir sicher, dass wir auf denselben Kanal eingestellt sind.
\end{tcolorbox}

\section*{Karte 1: Sachebene}
\begin{tcolorbox}[colback=calmgrey, colframe=calmblue, title={\textbf{Sachebene: Was sind die Fakten?}}]
\textbf{Beschreibung}: Ich frage mich, welche Information übermittelt wird. Was sind die reinen Fakten, ohne Interpretation? \newline
\textit{Stell dir vor}: Ich lese ein Buch mit klaren Fakten, wie eine Bedienungsanleitung für unseren Alltag. \newline
\textbf{Beispiel}: Mein Partner sagt: „Du hast den Abwasch nicht gemacht.“ Die Sachebene ist: Der Abwasch ist nicht erledigt. \newline
\textbf{Übung}: 
\begin{itemize}[leftmargin=*]
    \item Ich höre eine Aussage meines Partners (z. B. „Du bist zu spät“).
    \item Ich frage: „Was sind die Fakten?“ (z. B. Ich bin 10 Minuten später gekommen.)
    \item Ich schreibe die Fakten auf und bespreche sie im Jour Fixe.
\end{itemize}
\textbf{Tipp}: Ich bleibe ruhig und fokussiere mich auf die Fakten, um Missverständnisse zu vermeiden. \newline
\textit{Neurodiverse Anpassung}: Für PTBS (Trigger): Ich atme tief, bevor ich antworte. Für Autismus: Ich visualisiere die Fakten wie eine Checkliste.
\end{tcolorbox}

\section*{Karte 2: Beziehungsebene}
\begin{tcolorbox}[colback=calmgrey, colframe=calmblue, title={\textbf{Beziehungsebene: Wie ist unsere Verbindung?}}]
\textbf{Beschreibung}: Ich frage mich, was die Aussage über unsere Beziehung sagt. Fühle ich mich respektiert oder verletzt? \newline
\textit{Stell dir vor}: Unsere Beziehung ist wie ein Band, das wir durch Zuhören stärken. \newline
\textbf{Beispiel}: Mein Partner sagt: „Du hörst nie zu.“ Die Beziehungsebene ist: Ich fühle mich ignoriert. \newline
\textbf{Übung}: 
\begin{itemize}[leftmargin=*]
    \item Ich höre eine Aussage und frage: „Wie fühle ich mich dabei?“ (z. B. „Ich fühle mich nicht ernst genommen.“)
    \item Ich benutze das Gefühls-Wörterbuch, um mein Gefühl zu benennen (z. B. „Scham“).
    \item Ich teile mein Gefühl im Jour Fixe: „Ich fühle mich wie ein Schatten, wenn du sagst, ich höre nicht zu.“
\end{itemize}
\textbf{Tipp}: Ich validiere die Gefühle meines Partners, um Vertrauen aufzubauen. \newline
\textit{Neurodiverse Anpassung}: Für Borderline (intensive Emotionen): Ich atme, bevor ich reagiere. Für Autismus: Ich nutze klare Worte, um Gefühle zu beschreiben.
\end{tcolorbox}

\section*{Karte 3: Selbstoffenlegung}
\begin{tcolorbox}[colback=calmgrey, colframe=calmblue, title={\textbf{Selbstoffenlegung: Was sagt es über mich?}}]
\textbf{Beschreibung}: Ich frage mich, was meine Aussage über meinen Zustand verrät. Bin ich gestresst, traurig oder froh? \newline
\textit{Stell dir vor}: Ich öffne ein Fenster zu meinen Gefühlen, damit mein Partner mich versteht. \newline
\textbf{Beispiel}: Ich sage: „Du bist immer so laut.“ Die Selbstoffenlegung ist: Ich bin gestresst und überfordert. \newline
\textbf{Übung}: 
\begin{itemize}[leftmargin=*]
    \item Ich mache eine Aussage und frage: „Was sagt das über mich?“ (z. B. „Ich bin gestresst, weil ich viel zu tun habe.“)
    \item Ich benutze das Gefühls-Wörterbuch, um mein Gefühl zu benennen (z. B. „Angst“).
    \item Ich teile es im Jour Fixe: „Ich sage das, weil ich wie in einem Sturm bin.“
\end{itemize}
\textbf{Tipp}: Ich bin ehrlich über meine Gefühle, um Missverständnisse zu vermeiden. \newline
\textit{Neurodiverse Anpassung}: Für PTBS (Trigger): Ich benenne Trigger klar. Für ADHS: Ich halte meine Aussagen kurz.
\end{tcolorbox}

\section*{Karte 4: Appell}
\begin{tcolorbox}[colback=calmgrey, colframe=calmblue, title={\textbf{Appell: Was will ich erreichen?}}]
\textbf{Beschreibung}: Ich frage mich, was ich mit meiner Aussage erreichen will. Was brauche ich von meinem Partner? \newline
\textit{Stell dir vor}: Mein Appell ist wie ein Wegweiser, der zeigt, wohin ich will. \newline
\textbf{Beispiel}: Ich sage: „Kannst du leiser sein?“ Der Appell ist: Ich brauche Ruhe, um mich zu entspannen. \newline
\textbf{Übung}: 
\begin{itemize}[leftmargin=*]
    \item Ich mache eine Aussage und frage: „Was will ich?“ (z. B. „Ich möchte, dass wir zusammenarbeiten.“)
    \item Ich formuliere den Appell klar: „Bitte hilf mir beim Abwasch.“
    \item Ich bespreche ihn im Jour Fixe, um Vorwürfe zu vermeiden.
\end{itemize}
\textbf{Tipp}: Ich mache meinen Appell klar und freundlich, um Missverständnisse zu reduzieren. \newline
\textit{Neurodiverse Anpassung}: Für Borderline (Verbindung): Ich betone, dass wir ein Team sind. Für Autismus: Ich nutze konkrete Anweisungen.
\end{tcolorbox}

\section*{Anwendung}
Ich benutze diese Karten im Jour Fixe oder Alltag, wenn ein Missverständnis oder Vorwurf aufkommt. \newline
\textit{Stell dir vor}: Wir sind zwei Detektive, die mit diesen Karten einen Fall (Missverständnis) lösen. \newline
\textbf{Tipp}: Ich drucke die Karten aus (A6-Format) und lege sie auf den Tisch, um sie gemeinsam zu nutzen.

\end{document}
